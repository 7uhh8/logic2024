\documentclass[10pt,a4paper,oneside]{article}
\usepackage[utf8]{inputenc}
\usepackage[english,russian]{babel}
\usepackage{amsmath}
\usepackage{amsthm}
\usepackage{amssymb}
\usepackage{enumerate}
\usepackage{stmaryrd}
\usepackage{cmll}
\usepackage{mathrsfs}
\usepackage[left=2cm,right=2cm,top=2cm,bottom=2cm,bindingoffset=0cm]{geometry}
\usepackage{proof}
\usepackage{tikz}
\usepackage{multicol}
\usepackage{mathabx}
\usepackage{comment}
\usepackage{hyperref}
\usepackage[normalem]{ulem}

\makeatletter
\newcommand{\dotminus}{\mathbin{\text{\@dotminus}}}

\newcommand{\@dotminus}{%
  \ooalign{\hidewidth\raise1ex\hbox{.}\hidewidth\cr$\m@th-$\cr}%
}
\makeatother

\usetikzlibrary{arrows,backgrounds,patterns,matrix,shapes,fit,calc,shadows,plotmarks}

\newtheorem{definition}{Определение}
\newtheorem{theorem}{Теорема}
\begin{document}

\begin{center}{\Large\textsc{\textbf{Теоретические домашние задания}}}\\
             \it Математическая логика, ИТМО, М3232-М3239, осень 2024 года\end{center}

\section*{Задание №1. Знакомство с классическим исчислением высказываний.}

При решении заданий вам может потребоваться теорема о дедукции (будет доказана на второй лекции): 
\begin{theorem}
$\gamma_1,\dots,\gamma_n, \alpha \vdash \beta$ 
тогда и только тогда, когда $\gamma_1,\dots,\gamma_n \vdash \alpha\rightarrow\beta$. 
\end{theorem}

Пример использования: пусть необходимо доказать $\vdash A \rightarrow A$ --- то есть
доказать существование вывода формулы $A \rightarrow A$ (заметьте, так поставленное
условие не требует этот вывод предъявлять, только доказать его существование).
Тогда заметим, что последовательность из одной формулы $A$ доказывает $A \vdash A$. 
Далее, по теореме о дедукции, отсюда следует и $\vdash A \rightarrow A$ 
(то есть, существование вывода формулы $A \rightarrow A$, не использующего гипотезы).

Теорема будет доказана конструктивно: будет предъявлен алгоритм,
перестраивающий вывод $\gamma_1,\dots,\gamma_n, \alpha \vdash \beta$ 
в вывод $\gamma_1,\dots,\gamma_n \vdash \alpha\rightarrow\beta$ 

\begin{enumerate}
\item Докажите:
\begin{enumerate}
\item $\vdash (A \rightarrow A \rightarrow B) \rightarrow (A \rightarrow B)$
\item $\vdash \neg (A \with \neg A)$
\item $\vdash A \with B \rightarrow B \with A$
\item $\vdash A \vee B \rightarrow B \vee A$
\item $A \with \neg A \vdash B$
\end{enumerate}

\item Докажите:
\begin{enumerate}
\item $\vdash A \rightarrow \neg \neg A$
\item $\neg A, B \vdash \neg(A\& B)$
\item $\neg A,\neg B \vdash \neg( A\vee B)$
\item $ A,\neg B \vdash \neg( A\rightarrow B)$
\item $\neg A, B \vdash  A\rightarrow B$
\end{enumerate}

\item Докажите:
\begin{enumerate}
\item $\vdash (A \rightarrow B) \rightarrow (B \rightarrow C) \rightarrow (A \rightarrow C)$ 
\item $\vdash (A \rightarrow B) \rightarrow (\neg B \rightarrow \neg A)$ \emph{(правило контрапозиции)}
\item $\vdash \neg (\neg A \with \neg B) \rightarrow (A \vee B)$ \emph{(вариант I закона де Моргана)}
\item $\vdash A \vee B \rightarrow \neg(\neg A \with \neg B)$
\item $\vdash (\neg A \vee \neg B) \rightarrow \neg (A \with B)$ \emph{(II закон де Моргана)}
\item $\vdash (A \rightarrow B) \rightarrow (\neg A \vee B)$
\item $\vdash A \with B \rightarrow A \vee B$
\item $\vdash ((A \rightarrow B) \rightarrow A)\rightarrow A$ \emph{(закон Пирса)}
\item $\vdash A \vee \neg A$
\item $\vdash (A \with B \rightarrow C) \rightarrow (A \rightarrow B \rightarrow C)$
\item $\vdash A \with (B \vee C) \rightarrow (A \with B) \vee (A \with C)$ \emph{(дистрибутивность)}
\item $\vdash (A \rightarrow B \rightarrow C) \rightarrow (A \with B \rightarrow C)$
\item $\vdash (A \rightarrow B) \vee (B \rightarrow A)$
\item $\vdash (A \rightarrow B) \vee (B \rightarrow C) \vee (C \rightarrow A)$
\end{enumerate}

\item Даны высказывания $\alpha$ и $\beta$, причём $\vdash \alpha\rightarrow\beta$ и $\not\vdash\beta\rightarrow\alpha$. 
Укажите способ построения высказывания $\gamma$, такого, что
$\vdash\alpha\rightarrow\gamma$ и $\vdash\gamma\rightarrow\beta$, причём $\not\vdash\gamma\rightarrow\alpha$ и
$\not\vdash\beta\rightarrow\gamma$.

\item Покажите, что если $\alpha \vdash \beta$ и $\neg\alpha\vdash\beta$, то $\vdash\beta$.

\item Покажите, что классическое исчисление высказываний допускает правило Modus Tollens:
$$\infer{\neg\varphi}{\varphi\rightarrow\psi\quad\quad\neg\psi}$$

А именно, пусть дан некоторый вывод, в котором каждая формула --- либо аксиома, либо получена по правилу Modus Ponens, либо имеет вид $\delta_n \equiv \neg\varphi$, причём
ранее в доказательстве встречается $\delta_i \equiv \neg\psi$ и $\delta_j \equiv \varphi\rightarrow\psi$ (при этом $\max(i,j) < n$). Тогда такой вывод можно перестроить в
корректное доказательство в классическом исчислении высказываний.

В данном задании от вас требуется аккуратное изложение доказательства, видимо, использующее математическую индукцию. То есть, чётко
сформулированное индукционное предположение и полные доказательства базы и перехода.
\end{enumerate}

\end{document}
