\documentclass[10pt,a4paper,oneside]{article}
\usepackage[utf8]{inputenc}
\usepackage[english,russian]{babel}
\usepackage{amsmath}
\usepackage{amsthm}
\usepackage{amssymb}
\usepackage{enumerate}
\usepackage{stmaryrd}
\usepackage{cmll}
\usepackage{mathrsfs}
\usepackage[left=2cm,right=2cm,top=2cm,bottom=2cm,bindingoffset=0cm]{geometry}
\usepackage{proof}
\usepackage{tikz}
\usepackage{multicol}
\usepackage{mathabx}
\usepackage{comment}
\usepackage{hyperref}
\usepackage[normalem]{ulem}

\makeatletter
\newcommand{\dotminus}{\mathbin{\text{\@dotminus}}}

\newcommand{\@dotminus}{%
  \ooalign{\hidewidth\raise1ex\hbox{.}\hidewidth\cr$\m@th-$\cr}%
}
\makeatother

\usetikzlibrary{arrows,backgrounds,patterns,matrix,shapes,fit,calc,shadows,plotmarks}

\newtheorem{definition}{Определение}
\newtheorem{theorem}{Теорема}
\begin{document}

\begin{center}{\Large\textsc{\textbf{Теоретические домашние задания}}}\\
             \it Математическая логика, ИТМО, М3232-М3239, осень 2024 года\end{center}

\section*{Общие замечания}

\begin{enumerate}
\item Одно задание оценивается в 3.5 балла. При использовании TeX или Typst для оформления задание оценивается в 4 балла. При крайне плохом оформлении оценка может быть понижена до 3 баллов.
\item Заданием (по умолчанию) считается один пункт, занумерованный цифрой или буквой. Пункты без нумерации считаются частями одного задания.
\item Курс можно условно разделить на три части (исчисления высказываний и предикатов, формальная арифметика, теория множеств). В каждой из частей можно ответить не более четырёх заданий.
\end{enumerate}

\section*{Задание №1. Знакомство с классическим исчислением высказываний.}

При решении заданий вам может потребоваться теорема о дедукции (будет доказана на второй лекции): 
\begin{theorem}
$\gamma_1,\dots,\gamma_n, \alpha \vdash \beta$ 
тогда и только тогда, когда $\gamma_1,\dots,\gamma_n \vdash \alpha\rightarrow\beta$. 
\end{theorem}

Пример использования: пусть необходимо доказать $\vdash A \rightarrow A$ --- то есть
доказать существование вывода формулы $A \rightarrow A$ (заметьте, так поставленное
условие не требует этот вывод предъявлять, только доказать его существование).
Тогда заметим, что последовательность из одной формулы $A$ доказывает $A \vdash A$. 
Далее, по теореме о дедукции, отсюда следует и $\vdash A \rightarrow A$ 
(то есть, существование вывода формулы $A \rightarrow A$, не использующего гипотезы).

Теорема будет доказана конструктивно: будет предъявлен алгоритм,
перестраивающий вывод $\gamma_1,\dots,\gamma_n, \alpha \vdash \beta$ 
в вывод $\gamma_1,\dots,\gamma_n \vdash \alpha\rightarrow\beta$ 

\begin{enumerate}
\item Докажите:
\begin{enumerate}
\item $\vdash (A \rightarrow A \rightarrow B) \rightarrow (A \rightarrow B)$
\item $\vdash \neg (A \with \neg A)$
\item $\vdash A \with B \rightarrow B \with A$
\item $\vdash A \vee B \rightarrow B \vee A$
\item $A \with \neg A \vdash B$
\end{enumerate}

\item Докажите:
\begin{enumerate}
\item $\vdash A \rightarrow \neg \neg A$
\item $\neg A, B \vdash \neg(A\& B)$
\item $\neg A,\neg B \vdash \neg( A\vee B)$
\item $ A,\neg B \vdash \neg( A\rightarrow B)$
\item $\neg A, B \vdash  A\rightarrow B$
\end{enumerate}

\item Докажите:
\begin{enumerate}
\item $\vdash (A \rightarrow B) \rightarrow (B \rightarrow C) \rightarrow (A \rightarrow C)$ 
\item $\vdash (A \rightarrow B) \rightarrow (\neg B \rightarrow \neg A)$ \emph{(правило контрапозиции)}
\item $\vdash \neg (\neg A \with \neg B) \rightarrow (A \vee B)$ \emph{(вариант I закона де Моргана)}
\item $\vdash A \vee B \rightarrow \neg(\neg A \with \neg B)$
\item $\vdash (\neg A \vee \neg B) \rightarrow \neg (A \with B)$ \emph{(II закон де Моргана)}
\item $\vdash (A \rightarrow B) \rightarrow (\neg A \vee B)$
\item $\vdash A \with B \rightarrow A \vee B$
\item $\vdash ((A \rightarrow B) \rightarrow A)\rightarrow A$ \emph{(закон Пирса)}
\item $\vdash A \vee \neg A$
\item $\vdash (A \with B \rightarrow C) \rightarrow (A \rightarrow B \rightarrow C)$
\item $\vdash A \with (B \vee C) \rightarrow (A \with B) \vee (A \with C)$ \emph{(дистрибутивность)}
\item $\vdash (A \rightarrow B \rightarrow C) \rightarrow (A \with B \rightarrow C)$
\item $\vdash (A \rightarrow B) \vee (B \rightarrow A)$
\item $\vdash (A \rightarrow B) \vee (B \rightarrow C) \vee (C \rightarrow A)$
\end{enumerate}

\item Даны высказывания $\alpha$ и $\beta$, причём $\vdash \alpha\rightarrow\beta$ и $\not\vdash\beta\rightarrow\alpha$. 
Укажите способ построения высказывания $\gamma$, такого, что
$\vdash\alpha\rightarrow\gamma$ и $\vdash\gamma\rightarrow\beta$, причём $\not\vdash\gamma\rightarrow\alpha$ и
$\not\vdash\beta\rightarrow\gamma$.

\item Покажите, что если $\alpha \vdash \beta$ и $\neg\alpha\vdash\beta$, то $\vdash\beta$.

\item Покажите, что классическое исчисление высказываний допускает правило Modus Tollens:
$$\infer{\neg\varphi}{\varphi\rightarrow\psi\quad\quad\neg\psi}$$

А именно, пусть дан некоторый вывод, в котором каждая формула --- либо аксиома, либо получена по правилу Modus Ponens, либо имеет вид $\delta_n \equiv \neg\varphi$, причём
ранее в доказательстве встречается $\delta_i \equiv \neg\psi$ и $\delta_j \equiv \varphi\rightarrow\psi$ (при этом $\max(i,j) < n$). Тогда такой вывод можно перестроить в
корректное доказательство в классическом исчислении высказываний.

В данном задании от вас требуется аккуратное изложение доказательства, видимо, использующее математическую индукцию. То есть, чётко
сформулированное индукционное предположение и полные доказательства базы и перехода.
\end{enumerate}

\section*{Задание №2. Теоремы об исчислении высказываний. Знакомство с интуиционистским исчислением высказываний.}
\begin{enumerate}
\item Давайте вспомним, что импликация правоассоциативна: $\alpha\rightarrow\beta\rightarrow\gamma \equiv \alpha\rightarrow(\beta\rightarrow\gamma)$.
Но рассмотрим иную расстановку скобок: $(\alpha\rightarrow\beta)\rightarrow\gamma$. Возможно ли доказать логическое следствие
между этими вариантами расстановки скобок --- и каково его направление? Зависит ли это от варианта исчисления (классическое/интуиционистское)?
\item Покажите, что в классическом исчислении высказываний $\Gamma \models \alpha$ влечёт $\Gamma \vdash \alpha$.
\item Покажите, что в классическом исчислении высказываний $\Gamma \vdash \alpha$ влечёт $\Gamma \models \alpha$.
\item Возможно ли, что какая-то из аксиом задаётся двумя разными схемами аксиом? Опишите все возможные коллизии для какой-то одной такой пары схем аксиом.
Ответ обоснуйте (да, тут потребуется доказательство по индукции).

\item Заметим, что можно вместо отрицания ввести в исчисление ложь. Рассмотрим \emph{исчисление высказываний с ложью}.
В этом языке будет отсутствовать одноместная связка $(\neg)$, вместо неё будет присутствовать нульместная
связка <<ложь>> $(\bot)$, а 9 и 10 схемы аксиом будут заменены на одну схему:

\begin{tabular}{ll}
$(9_\bot)$ & $((\alpha\rightarrow\bot)\rightarrow\bot)\rightarrow\alpha$
\end{tabular}

Будем записывать доказуемость в новом исчислении как $\vdash_\bot \alpha$, а доказуемость в исчислении высказываний
с отрицанием как $\vdash_\neg \beta$. Также определим операцию трансляции между языками обычного исчисления высказываний и исчисления с ложью
как операции рекурсивной замены $\bot := A \with \neg A$ и $\neg \alpha := \alpha \rightarrow \bot$ (и обозначим их
как $|\varphi|_\neg$ и $|\psi|_\bot$ соответственно).

Докажите:
\begin{enumerate}
\item $\vdash_\bot \alpha$ влечёт $\vdash_\neg |\alpha|_\neg$
\item $\vdash_\neg \alpha$ влечёт $\vdash_\bot |\alpha|_\bot$
\end{enumerate}

\item Покажите, что топологическое пространство на вещественных числах с базой $\mathcal{B} = \{(a,b)\ |\ a,b \in \mathbb{R}\}$ совпадает
с топологическим пространством $\mathbb{R}$ из матанализа (то есть, совпадают множества открытых множеств).

\item Покажите, что дискретная топология, антидискретная топология (открыты только $\varnothing$ и $X$), топология стрелки, топология Зарисского
(носитель --- $\mathbb{R}$, открыты $\varnothing$, $\mathbb{R}$ и все множества с конечным дополнением) являются топологиями.

\item Заметим, что определения стараются давать как можно более узкими: если некоторое свойство вытекает из других, то это уже не свойство из определения, а теорема.
Поэтому приведите примеры $\langle X, \Omega\rangle$, нарушающие только первое, только второе и только третье условие на топологию.

\item Напомним, что замкнутое множество --- такое, дополнение которого открыто.
Заметим, что на $\mathbb{R}$ ровно два множества одновременно открыты и замкнуты --- $\varnothing$ и всё пространство. Постройте другую
(не евклидову) топологию на $\mathbb{R}$, чтобы в ней было ровно четыре множества, которые одновременно открыты и замкнуты. А возможно ли построить
топологическое пространство, в котором было бы ровно три открыто-замкнутых множества?

\item Назовём минимальной базой топологии такую базу, что в ней никакое множество не может быть получено объединением семейства других множеств из базы. 
\begin{enumerate}
\item Постройте минимальную базу для дискретной топологии.
\item Существует ли минимальная база для топологии стрелки?
\item Существует ли минимальная база для топологии Зарисского (носитель --- $\mathbb{R}$, открыты $\varnothing$, $\mathbb{R}$ и все множества с конечным дополнением)?
\end{enumerate}

\item Предложите пример топологического пространства, в котором пересечение произвольного семейства открытых множеств --- открыто.
Топологическое пространство должно иметь бесконечный носитель (чтобы задача имела содержательный смысл) и не должно иметь дискретную 
или антидискретную топологию (не должно быть в каком-то смысле вырожденным).

\item Наибольшим (наименьшим) значением в каком-то множестве назовём такое, которое больше (меньше) всех других элементов множества. 
Несложно заметить, что для отношения включения множеств далеко всегда такое можно определить: например, на $\mathbb{R}^2$ не существует наибольшего круга с радиусом 1,
хотя такой круг существует на $\{ z\ |\ z \in \mathbb{R}^2, |z| \le 1\}$.

\emph{Внутренностью} множества $A^\circ$ назовём наибольшее открытое множество, содержащееся в $A$. 
Покажите, что внутренность множества всегда определена.

\item Напомним определения: \emph{замкнутое} множество --- такое, дополнение которого открыто.
\emph{Замыканием} множества $\overline{A}$ назовём наименьшее замкнутое множество, содержащее $A$.
Назовём \emph{окрестностью} точки $x$ такое открытое множество $V$, что $x \in V$.
Будем говорить, что точка $x \in A$ \emph{внутренняя}, если существует окрестность $V$, что $V \subseteq A$.
Точка $x$ --- \emph{граничная}, если любая её окрестность $V$ пересекается как с $A$, так и с его дополнением.
\begin{enumerate}
\item \begin{itemize} \item Покажите, что $A$ открыто тогда и только тогда, когда все точки $A$ --- внутренние.
Также покажите, что $A^\circ = \{ x|x \in A \with x\text{ --- внутренняя точка}\}$;
\item Покажите, что $A$ замкнуто тогда и только тогда, когда содержит все свои граничные точки.
Также покажите, что $\overline{A} = \{ x\ |\ x\text{ --- внутренняя или граничная точка}\}$.
\item Верно ли, что $\overline{A} = X \setminus ((X\setminus A)^\circ)$?\end{itemize}
\item Пусть $A \subseteq B$. Как связаны $A^\circ$ и $B^\circ$, а также $\overline{A}$ и $\overline{B}$?
 Верно ли $(A \cap B)^\circ = A^\circ \cap B^\circ$ и $(A \cup B)^\circ = A^\circ \cup B^\circ$?
\item \emph{Задача Куратовского.} Будем применять операции взятия внутренности и замыкания к некоторому множеству
всевозможными способами. Сколько различных множеств может всего получиться?
\emph{Указание.} Покажите, что $\overline{\left(\overline{A^\circ}\right)^\circ} = \overline{A^\circ}$.
\end{enumerate}

\item Задача проверки высказываний на истинность в ИИВ сложнее, чем в КИВ --- не существует конечного набора значений,
на которых можно проверить формулу, чтобы определить её истинность (мы эту теорему докажем). Тем не менее, если формула
опровергается, то она опровергается на $\mathbb{R}$ с евклидовой топологией. Если же такого опровержения нет, то формула 
доказуема (то есть, ИИВ семантически полно на $\mathbb{R}$). Например, формула $A \vee \neg A$ опровергается при $\llbracket A \rrbracket = (0,+\infty)$,
так как $\llbracket A \vee \neg A \rrbracket = \mathbb{R}\setminus\{0\}$.

Очевидно, что любая интуиционистская тавтология общезначима и в классической логике:
\begin{itemize}
\item формула общезначима в интуиционистской логике; 
\item значит, истинна при всех оценках; 
\item значит, в частности, при всех оценках на $\mathbb{R}$;
\item то есть, по теореме, упомянутой выше, доказуема в ИИВ;
\item а схема аксиом 10и --- частный случай схемы аксиом 10.
\end{itemize}
Обратное же неверно. Определите, являются ли следующие формулы тавтологиями в КИВ и ИИВ (предложите опровержение или доказательство общезначимости/выводимости для каждого
из исчислений):

\begin{enumerate}
\item $((A \rightarrow B) \rightarrow A) \rightarrow A$;
\item $\neg\neg A \rightarrow A$;
\item $(A \rightarrow B) \vee (B \rightarrow A)$ (из двух утверждений одно непременно следует из другого: например, <<я не люблю зиму>> и <<я не люблю лето>>);
\item $(A \rightarrow B) \vee (B \rightarrow C)$;
\item $(A \rightarrow (B \vee \neg B)) \vee (\neg A \rightarrow (B \vee \neg B))$;
\item $\alpha\vee\beta \vdash \neg(\neg\alpha\with\neg\beta)$ и $\neg(\neg\alpha\with\neg\beta) \vdash \alpha\vee\beta$;
\item $\neg\alpha\with\neg\beta \vdash \neg(\alpha\vee\beta)$ и $\neg(\alpha\vee\beta) \vdash \neg\alpha\with\neg\beta$;
\item $\alpha\rightarrow\beta \vdash \neg\alpha\vee\beta$ и $\neg\alpha\vee\beta \vdash \alpha\rightarrow\beta$.
\end{enumerate}

\item Известно, что в КИВ все связки могут быть выражены через операцию <<и-не>> (<<или-не>>). Также, они могут быть выражены друг через
друга (достаточно, например, отрицания и конъюнкции). Однако, в ИИВ это не так.

Покажите, что никакие связки не выражаются друг через друга: то есть, нет такой формулы $\varphi(A,B)$ из языка 
интуиционистской логики, не использующей связку $\star$, что $\vdash A \star B \rightarrow \varphi(A,B)$ и $\vdash\varphi(A,B) \rightarrow A \star B$.
Покажите это для каждой связки в отдельности:
\begin{enumerate}
\item конъюнкция;
\item дизъюнкция;
\item импликация;
\item отрицание.
\end{enumerate}

\end{enumerate}

\section*{Задание №3. Изоморфизм Карри-Ховарда. Дополнительные топологические определения. Решётки. }
\begin{enumerate}
\item Непрерывной функцией называется такая, для которой прообраз открытого множества всегда открыт.
Путём на топологическом пространстве $X$ назовём непрерывное отображение вещественного отрезка $[0,1]$ в $X$.
Опишите пути (то есть, опишите, какие функции могли бы являться путями):
\begin{enumerate}
\item на $\mathbb{N}$ (с дискретной топологией);
\item в топологии Зарисского;
\item на дереве (с топологией с лекции);
\end{enumerate}

\item Докажите, что функция $f: \mathbb{R}\rightarrow\mathbb{R}$ непрерывна тогда и только тогда, когда
$\lim_{x \rightarrow x_0} f(x) = f(x_0)$ для всех $x_0 \in \mathbb{R}$.

\item Связным множеством в топологическом пространстве назовём такое, которое связно как подпространство.
Линейно связным множеством назовём такое, в котором две произвольные точки могут быть соединены путём,
образ которого целиком лежит в множестве. 
\begin{enumerate}
\item Покажите, что линейно связное множество всегда связно;
\item Покажите, что связное не обязательно линейно связное.
\end{enumerate}

\item Всегда ли непрерывным образом связного пространства является другое связное (под)пространство? Докажите или опровергните.

\item Как мы помним с лекции, возможно доказывать интуиционистские утверждения, воспользовавшись изоморфизмом Карри-Ховарда, 
то есть написав соответствующую программу на каком-нибудь статически типизированном языке программирования. 

Например, на C++ так можно доказать $A\rightarrow A$:
\begin{verbatim}
A identity (A x) { return x; }
\end{verbatim}

Докажите следующие утверждения, не пользуясь в коде тем фактом, что обычно языки программирования
противоречивы (то есть, не используйте исключений, функций, не возвращающих управления, и других подобных конструкций).
\begin{enumerate}
\item $A \rightarrow B \rightarrow A$
\item $A \with B \rightarrow A \vee B$
\item $(A \with (B \vee C)) \rightarrow ((A \with B) \vee (A \with C))$
\item $(A \rightarrow C) \with (B \rightarrow C) \with ((A \vee B) \rightarrow C)$
\item $(B \vee C \rightarrow A) \rightarrow (B \rightarrow A) \with (C \rightarrow A)$
\item $(A \rightarrow B) \rightarrow (\neg B \rightarrow \neg A)$
\item $((A \rightarrow B) \rightarrow C) \rightarrow (A \rightarrow (B \rightarrow C))$
\item $(A \rightarrow B) \with (A \rightarrow \neg B) \rightarrow \neg A$
\item Выразимые в интуиционистском исчислении высказываний аналоги правил де Моргана для импликации.
\end{enumerate}

\item Рассмотрим подмножество частично упорядоченного множества, и рассмотрим следующие свойства:
(а) наличие наибольшего элемента; (б) наличие супремума;
(в) наличие единственного максимального элемента. Всего можно рассмотреть шесть утверждений ((а) влечёт (б), 
(а) влечёт (в), и т.п.) --- про каждое определите, выполнено ли оно в общем случае,
и приведите либо доказательство, либо контрпример. Задача состоит из одного пункта, для получения баллов 
все шесть утверждений должны быть разобраны.

\item Покажите следующие утверждения для импликативных решёток:
\begin{enumerate}
\item монотонность: пусть $a \preceq b$ и $c \preceq d$, тогда $a + c \preceq b + d$ и $a \cdot c \preceq b \cdot d$;
\item \emph{законы поглощения:} $a \cdot (a + b) = a$; $a + (a \cdot b) = a$;
\item $a \preceq b$ выполнено тогда и только тогда, когда $a \rightarrow b = 1$;
\item из $a \preceq b$ следует $b\rightarrow c \preceq a\rightarrow c$ и $c\rightarrow a \preceq c \rightarrow b$;
\item из $a \preceq b \rightarrow c$ следует $a \cdot b \preceq c$;
\item $b \preceq a \rightarrow b$ и $a \rightarrow (b \rightarrow a) = 1$;
\item $a \rightarrow b \preceq ((a \rightarrow (b \rightarrow c)) \rightarrow (a \rightarrow c))$;
\item $a \preceq b \rightarrow a \cdot b$ и $a \rightarrow (b \rightarrow (a \cdot b)) = 1$
\item $a \rightarrow c \preceq (b \rightarrow c) \rightarrow (a + b \rightarrow c)$
\item импликативная решётка дистрибутивна: $(a + b) \cdot c = (a \cdot c) + (b \cdot c)$
\end{enumerate}

\item Докажите, основываясь на формулах предыдущих заданий, что интуиционистское исчисление высказываний
корректно, если в качестве модели выбрать алгебру Гейтинга.

\item Покажите, что на конечном множестве дистрибутивная решётка всегда импликативна.
\item Постройте пример дистрибутивной, но не импликативной решётки.
\item Покажите, что в дистрибутивной решётке всегда $a + (b \cdot c) = (a + b) \cdot (a + c)$.

\item Пусть $R \subseteq A\times A$ --- отношение эквивалентности (то есть транзитивное, рефлексивное и симметричное).
Тогда фактор-множество $A/_R := \{ [ x ]_R\ |\ x \in A\}$ --- множество \emph{классов эквивалентности},
где $[x]_R = \{ t\in R\ |\ tRx\}$.

Покажите, что каждый элемент множества $A$ принадлежит в точности одному классу эквивалентности. Два класса эквивалентности либо не пересекаются, либо совпадают.

\item Пусть $R \subseteq A\times A$ --- отношение нестрогого предпорядка (транзитивное и рефлексивное). И пусть
$a \approx b$, если $aRb$ и $bRa$. Покажите, что 
\begin{enumerate}
\item Если $aRb$ и $a \approx a'$, $b \approx b'$, то $a'Rb'$.
\item $R/_\approx$ --- отношение нестрогого порядка на $A/_\approx$ в следующем смысле: $[a]_\approx R/_\approx [b]_\approx$ выполнено, если $aRb$
(корректность определения также необходимо показать).
\end{enumerate}

\item Покажите, что $(\preceq)$ из определения алгебры Линденбаума --- отношение нестрогого предпорядка, $(\approx)$ --- отношение эквивалентности, а
$(\preceq)/_\approx$ --- отношение нестрогого порядка.

\item Покажите, что $[\alpha]_\mathcal{L} + [\beta]_\mathcal{L} = [\alpha\vee\beta]_\mathcal{L}$.
Зависит ли результат от выбора представителей классов эквивалентности $[\alpha]$ и $[\beta]$? Ответ также докажите.
\item Покажите, что $[\alpha\rightarrow\beta]_\mathcal{L}$ --- псевдодополнение $[\alpha]_\mathcal{L}$ до $[\beta]_\mathcal{L}$.
\end{enumerate}

\end{document}
