\documentclass[11pt,a4paper,oneside]{scrartcl}
\usepackage[utf8]{inputenc}
\usepackage[english,russian]{babel}
\usepackage[top=1cm,bottom=1cm,left=1cm,right=1cm]{geometry}

\begin{document}
\pagestyle{empty}

\begin{center}
{\large\scshape\bfseries Программа курса <<Математическая логика>>}\\
{\large\scshape Вопросы к первому коллоквиуму.}\\
\itshape ИТМО, группы M3232--M3239, осень 2024 г.
\end{center}

%\vspace{0.3cm}

\begin{enumerate}
\item Исчисление высказываний:
\begin{enumerate}
\item Предметный язык и язык исследователя (метаязык). Соглашения об обозначениях. Схемы формул.
\item Язык исчисления высказываний.
\item Оценка высказываний, общезначимость, следование.
\item Доказуемость, гипотезы (контекст), выводимость.
\item Корректность, полнота, противоречивость и непротиворечивость (эквивалентные формулировки).
\item Теорема о дедукции для исчисления высказываний (формулировка). Теорема о полноте исчисления высказываний (формулировка).
\end{enumerate}
\item Топологическое пространство
\begin{enumerate}
\item Определение.
\item Метрическое пространство.
\item Примеры (топология стрелки, Зарисского, топология на деревьях). 
\item Открытые и замкнутые множества. Связность. Компактность. 
\item Непрерывные функции. Путь. Линейная связность. 
\end{enumerate}
\item Интуиционистское исчисление высказываний:
\begin{enumerate}
\item Доказательства чистого существования.
\item BHK-интерпретация. 
\item Решётки. 
\item Дистрибутивная решётка.
\item Булевы и псевдобулевы алгебры.
\item Алгебра Линденбаума. 
\item Полнота интуиционистского исчисления высказываний в псевдобулевых алгебрах (формулировка, идея доказательства).
\item Модели Крипке. Вынужденность.
\item Сведение моделей Крипке к псевдобулевым алгебрам. 
\item Нетабличность ИИВ (формулировка теоремы).
\end{enumerate}
\item Дизъюнктивность интуиционистского исчисления высказываний.
\begin{enumerate}
\item Гёделева алгебра. Операция $\Gamma(A)$.
\item Дизъюнктивность ИИВ (формулировка).
\end{enumerate}
\item Разрешимость интуиционистского исчисления высказываний (формулировка).
\item Исчисление предикатов.
\begin{enumerate}
\item Категорический силлогизм: предикат, субъект, средний термин, фигуры
\item Соотношения между терминами (A,E,I,O), модусы, модус Barbara.
\item Модусы: сильные, слабые, <<плохие>>, приведите по примеру каждого модуса (с указанием мнемонического имени).
\item Язык исчисления предикатов.
\item Сокращения метаязыка для исчисления предикатов.
\item Cледование в исчислении предикатов.
\item Теорема о дедукции в исчислении предикатов (формулировка).
\item Теорема о корректности исчисления предикатов (формулировка).
\end{enumerate}

\end{enumerate}
\end{document}
