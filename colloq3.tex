\documentclass[11pt,a4paper,oneside]{scrartcl}
\usepackage[utf8]{inputenc}
\usepackage[english,russian]{babel}
\usepackage[top=1cm,bottom=1cm,left=1cm,right=1cm]{geometry}

\begin{document}
\pagestyle{empty}

\begin{center}
{\large\scshape\bfseries Программа курса <<Математическая логика>>}\\
{\large\scshape Вопросы к третьему коллоквиуму.}\\
\itshape ИТМО, группы M3232--M3239, осень 2024 г.
\end{center}

%\vspace{0.3cm}

\begin{enumerate}
\item Лямбда-исчисление. Пред-лямбда-термы и лямбда-термы. Альфа-эквивалентность, бета-редукция
и бета-эквивалентность. Теорема Чёрча-Россера. 
Комбинатор неподвижной точки. Комбинаторный базис $SK$.
Истина и ложь. Чёрчевские нумералы. 
\item Натуральный вывод. Импликативный фрагмент интуиционистского исчисления высказываний.
Теорема о замкнутости доказуемости И.ф.И.И.В.
Просто-типизированное лямбда исчисление. Изоморфизм Карри-Ховарда 
(высказывание, доказательство, импликация, конъюнкция, дизъюнкция, ложь). 
\item Теория множеств. Определения равенства. Парадокс брадобрея. Аксиоматика Цермело-Френкеля. 
Конструктивные аксиомы
(пустого, пары, объединения, множества подмножеств, выделения).
Частичный, линейный, полный порядок. Ординальные числа, аксиома бесконечности. 
\item Конечные ординалы, существование ординала $\omega$, операции над ординалами, 
доказательство $1+\omega\ne\omega+1$. Связь ординалов и упорядочений. Аксиомы фундирования и подстановки.
\item Кардинальные числа, мощность множеств. Теорема Кантора-Бернштейна, теорема Кантора. 
\item Мощность модели. Элементарные подмодели. Теорема Лёвенгейма-Сколема, парадокс Сколема.
\item Аксиома выбора, альтернативные формулировки (лемма Цорна, теорема Цермело, существование
частичной обратной), доказательство переходов (кроме доказательства леммы Цорна).
\item Применение аксиомы выбора: эквивалентность определений пределов (по Коши и по Гейне).
Теорема Диаконеску. Ослабленные варианты (счётный выбор и зависимый выбор), универсум фон-Неймана.
Аксиома конструктивности.
\item 
(в зависимости от даты коллоквиума, вопросы по пункту могут отсутствовать)
Индукция и полная индукция. Трансфинитная индукция. Применение трансфинитной индукции.
Система $S_\infty$. 
Сечение, устранение сечений. Доказательство непротиворечивости формальной арифметики.
\end{enumerate}
\end{document}
