\documentclass[aspectratio=169]{beamer}
\usepackage[utf8]{inputenc}
\usepackage[english,russian]{babel}
\usepackage{cancel}
\usepackage{amssymb}
\usepackage{stmaryrd}
\usepackage{cmll}
\usepackage{graphicx}
\usepackage{amsthm}
\usepackage{tikz}
\usepackage{multicol}
\usetikzlibrary{patterns}
\usepackage{chronosys}
\usepackage{proof}
\usepackage{multirow}
\setbeamertemplate{navigation symbols}{}
%\usetheme{Warsaw}

\newtheorem{thm}{Теорема}[section]
\newtheorem{dfn}{Определение}[section]
\newtheorem{lmm}{Лемма}[section]
\newtheorem{exm}{Пример}[section]
\newtheorem{snote}{Пояснение}[section]

\newcommand{\divisible}%
{\mathrel{\lower.2ex%
\vbox{\baselineskip=0.7ex\lineskiplimit=0pt%
\kern6pt \hbox{.}\hbox{.}\hbox{.}}%
}}

\begin{document}

\newcommand\doubleplus{+\kern-1.3ex+\kern0.8ex}
\newcommand\mdoubleplus{\ensuremath{\mathbin{+\mkern-10mu+}}}

\begin{frame}[fragile]{История лямбда-исчисления}
\begin{itemize}
\item Фреге, 1893. Все функции --- одноместные. Сложение: $a+b = (+_a)(b)$, одноместная
функция, возвращающая другую одноместную функцию:

$(+_5)(4) = 9$, $(+_5)((+_3) 7) = 15$

\item Анонимные функции. Что такое, например, возведение в квадрат? 
\begin{itemize}
\item Возведение в квадрат \emph{чего}? $f(x) =x^2$. 
\item А зачем тут $f$ и почему $x$?
\item Давайте напишем это как-то так: $\hat{x}^2$ (Principa Mathematica, 1910)
\item От имени толк бывает: в Фортране переменные \verb!I..N! --- целые, остальные --- плавающие.
\end{itemize}
\item Алонзо Чёрч, 1930+ --- попытка построить исчисление для матлогики

Последовательное превращение записи: $\wedge x.x^2$, $\lambda x.x^2$. 
Точка --- тоже из Principa Mathematica: $(\exists a).\varphi(a)$.

\item Тезис Чёрча сформулирован впервые про лямбда-исчисление
\item 1936 --- лямбда-исчисление в современном варианте (для программирования)
\item 1940 --- просто-типизированное лямбда-исчисление 
\end{itemize}
\end{frame}

\begin{frame}{Лямбда-исчисление, синтаксис}
$$\Lambda ::= (\lambda x.\Lambda) | (\Lambda\ \Lambda) | x$$

Мета-язык: 
\begin{itemize}
\item Мета-переменные:\begin{itemize}
\item $A\dots Z$ --- мета-переменные для термов. 
\item $x,y,z$ --- мета-переменные для переменных. 
\end{itemize}

\item Правила расстановки скобок аналогичны правилам для кванторов:
\begin{itemize}
\item Лямбда-выражение ест всё до конца строки
\item Аппликация левоассоциативна
\end{itemize}
\end{itemize}

\begin{exm}
\begin{itemize}
\item $a\ b\ c\ (\lambda d.e\ f\ \lambda g.h)\ i \equiv \Big({\color{red}\Big(}((a\ b)\ c)\ {\color{blue}\Big(}\lambda d.((e\ f)\ (\lambda g.h)){\color{blue}\Big)}{\color{red}\Big)}\ i\Big)$
\item $0 := \lambda f.\lambda x.x;\quad(+1) := \lambda n.\lambda f.\lambda x.n\ f\ (f\ x);\quad(+2) := \lambda x.(+1)\ ((+1)\ x)$
\end{itemize}
\end{exm}
\end{frame}

\begin{frame}{Альфа-эквивалентность}
$$FV(A) = \left\{\begin{array}{ll} \{x\}, & A \equiv x\\
  FV(P)\cup FV(Q), & A \equiv P\ Q\\
  FV(P)\setminus\{x\}, & A \equiv \lambda x.P\end{array}\right.$$

Примеры: 
\begin{itemize}
\item $M := \lambda b.\lambda c.a\ c\ (b\ c)$; $FV(M) = \{a\}$
\item $N := x\ (\lambda x.(x\ (\lambda y.x)))$; $FV(N) = \{x\}$
\end{itemize}

\begin{dfn}$A=_\alpha B$, если и только если выполнено одно из трёх:
\begin{enumerate}
\item $A \equiv x$, $B \equiv y$, $x \equiv y$;
\item $A \equiv P_a Q_a$, $B \equiv P_b Q_b$ и $P_a =_\alpha P_b$, $Q_a =_\alpha Q_b$;
\item $A \equiv (\lambda x.P)$, $B \equiv (\lambda y.Q)$, $P[x := t] =_\alpha Q[y := t]$, где $t$ не входит в $A$ и $B$.
\end{enumerate}\end{dfn}

\begin{dfn}$L = \Lambda/_{=_\alpha}$\end{dfn}
\end{frame}

\begin{frame}{Альфа-эквивалентность, пример}
\begin{enumerate}
\item \color{gray}$A \equiv x$, $B \equiv y$, $x \equiv y$;
\item \color{gray}$A \equiv P_a Q_a$, $B \equiv P_b Q_b$ и $P_a =_\alpha P_b$, $Q_a =_\alpha Q_b$;
\item \color{gray}$A \equiv (\lambda x.P)$, $B \equiv (\lambda y.Q)$, $P[x := t] =_\alpha Q[y := t]$, где $t$ не входит в $A$ и $B$.
\end{enumerate}\vspace{-0.2cm}\begin{lmm}\vspace{-0.3cm}
$$\lambda a.\lambda b.a\ b =_\alpha \lambda b.\lambda a.b\ a$$\vspace{-0.3cm}
\end{lmm}\vspace{-0.3cm}
\begin{proof}
\begin{center}\begin{tabular}{rcll}
  $t$ & $=_\alpha$ &$t$& Правило 1\\
  $s$ & $=_\alpha$ &$s$& Правило 1\\
  $t\ s$ & $=_\alpha$ &$t\ s$& Правило 2\\
  $\lambda b.(t\ b)$ & $=_\alpha$ &$\lambda a.(t\ a)$ & Правило 3\\
  $\lambda a.\lambda b.(a\ b)$ & $=_\alpha$ &$\lambda b.\lambda a.(b\ a)$ & Правило 3\\
\end{tabular}\end{center}\vspace{-0.3cm}
\end{proof}
\end{frame}

\begin{frame}{Бета-редукция}
Интуиция: вызов функции.
\begin{center}\begin{tabular}{l|l}
$\lambda$-выражение & Python \\\hline
$\lambda f.\lambda x.f\ x$ & \texttt{def one(f,x): return f(x)}\\
$(\lambda x.x\ x)\ (\lambda x.x\ x)$ & \texttt{(lambda x: x x) (lambda x: x x)}\\
 &   \texttt{def omega(x): return x(x); omega(omega)}
\end{tabular}\end{center}

\pause
\begin{dfn} Терм вида $(\lambda x.P)\ Q$ --- бета-редекс.\end{dfn}
\begin{dfn} $A \rightarrow_\beta B$, если:
\begin{enumerate}
\item $A \equiv (\lambda x.P)\ Q$, $B \equiv P\ [x := Q]$, при условии свободы для подстановки;
\item $A \equiv (P\ Q)$, $B \equiv (P'\ Q')$, при этом $P \rightarrow_\beta P'$ и $Q = Q'$, либо $P = P'$ и $Q \rightarrow_\beta Q'$;
\item $A \equiv (\lambda x.P)$, $B \equiv (\lambda x.P')$, и $P \rightarrow_\beta P'$.
\end{enumerate}
\end{dfn}\end{frame}

\begin{frame}{Бета-редукция, пример}
\begin{exm}
$(\lambda x.x\ x)\ (\lambda n.n) \rightarrow_\beta (\lambda n.n)\ (\lambda n.n) \rightarrow_\beta \lambda n.n$
\end{exm}
\begin{exm}
$(\lambda x.x\ x)\ (\lambda x.x\ x) \rightarrow_\beta (\lambda x.x\ x)\ (\lambda x.x\ x)$
\end{exm}
\end{frame}

\begin{frame}{Нормальная форма}
\begin{dfn}Лямбда-терм $N$ находится в нормальной форме, если нет $Q$: $N \rightarrow_\beta Q$.\end{dfn}
\begin{exm}В нормальной форме:\\
$\lambda f.\lambda x.x\ (f\ (f\ \lambda g.x))$\end{exm}\pause
\begin{exm}Не в нормальной форме (редексы подчёркнуты):\\
$\lambda f.\lambda x.\underline{(\lambda g.x)\ (f\ (f\ x))}$\\
$(\underline{(\lambda x.x)\ (\lambda x.x)})\ (\underline{(\lambda x.x)\ (\lambda x.x)})$
\end{exm}
\begin{dfn}$(\twoheadrightarrow_\beta)$ --- транзитивное и рефлексивное замыкание $(\rightarrow_\beta)$.\end{dfn}
\end{frame}

\begin{frame}{Булевские значения}
$T := \lambda x.\lambda y.x$
$F := \lambda x.\lambda y.y$

Тогда: $Or := \lambda a.\lambda b.a\ T\ b$:

$Or\ F\ T = \underline{((\lambda a.\lambda b.a\ T\ b)\ F)}\ T \rightarrow_\beta (\lambda b.F\ T\ b)\ T
\rightarrow_\beta F\ T\ T =$

$=(\lambda x.\lambda y.y)\ T\ T\rightarrow_\beta (\lambda y.y)\ T \rightarrow_\beta T$
\end{frame}

\begin{frame}{Чёрчевские нумералы}
$$f^{(n)}(x) = \left\{\begin{array}{ll}x, & n = 0\\f(f^{(n-1)}(x)), & n > 0\end{array}\right.$$

\begin{dfn}
Чёрчевский нумерал $\overline{n} = \lambda f.\lambda x.f^{(n)}(x)$
\end{dfn}

\begin{exm}
$\overline{3} = \lambda f.\lambda x.f(f(f(x)))$

Инкремент: $Inc = \lambda n.\lambda f.\lambda x.n\ f\ (f\ x)$
\end{exm}\vspace{-0.3cm}

$$\begin{array}{l}(\lambda n.\lambda f.\lambda x

.n\ f\ (f\ x))\ \overline{0} = (\lambda n.\lambda f.\lambda x.n\ f\ (f\ x))\ (\lambda f'.\lambda x'.x') \rightarrow_\beta \pause\\
  \dots\lambda f.\lambda x.(\lambda f'.\lambda x'.x')\ f\ (f\ x) \rightarrow_\beta \pause\\
  \dots\lambda f.\lambda x.(\lambda x'.x')\ (f\ x) \rightarrow_\beta \pause\\
  \dots\lambda f.\lambda x.f\ x = \overline{1}\end{array}$$
\pause

Декремент: $Dec = \lambda n.\lambda f.\lambda x.n\ (\lambda g.\lambda h.h\ (g\ f))\ (\lambda u.x)\ (\lambda u.u)$
\end{frame}

\begin{frame}{Упорядоченная пара и алгебраический тип}
\begin{dfn}$Pair(a,b) := \lambda s.s\ a\ b$\\
$Fst := \lambda p.p\ T$\\
$Snd := \lambda p.p\ F$
\end{dfn}

\begin{exm}
$Fst (Pair (a,b)) = (\lambda p.p\ T)\ \lambda s.s\ a\ b \twoheadrightarrow_\beta (\lambda s.s\ a\ b)\ T \twoheadrightarrow_\beta a$
\end{exm}

\begin{dfn}
$InL\ L := \lambda p.\lambda q.p\ L$\\
$InR\ R := \lambda p.\lambda q.q\ R$\\
$Case\ t\ f\ g := t\ f\ g$
\end{dfn}

\end{frame}

\begin{frame}{Теорема Чёрча-Россера}
\begin{thm}[Чёрча-Россера] Для любых термов $N$, $P$, $Q$, если $N \twoheadrightarrow_\beta P$, $N \twoheadrightarrow_\beta Q$,
и $P \ne Q$, то найдётся $T$: $P \twoheadrightarrow_\beta T$ и $Q \twoheadrightarrow_\beta T$.\end{thm}
\begin{thm}Если у терма $N$ существует нормальная форма, то она единственна\end{thm}
\begin{proof}Пусть не так и $N \twoheadrightarrow_\beta P$ вместе с $N \twoheadrightarrow_\beta Q$, $P \ne Q$.
Тогда по теореме Чёрча-Россера существует $T$: $P \twoheadrightarrow_\beta T$ и $Q \twoheadrightarrow_\beta T$,
причём $T \ne P$ или $T \ne Q$ в силу транзитивности $(\twoheadrightarrow_\beta)$\end{proof}
\end{frame}

\begin{frame}{Бета-эквивалентность, неподвижная точка}
\begin{exm}$\Omega = (\lambda x.x\ x)\ (\lambda x.x\ x)$ не имеет нормальной формы:
$\Omega \rightarrow_\beta \Omega$\end{exm}
\begin{dfn}$(=_\beta)$ --- транзитивное, рефлексивное и симметричное замыкание $(\rightarrow_\beta)$.\end{dfn}
\begin{thm}Для любого терма $N$ найдётся такой терм $R$, что $R =_\beta N\ R$.\end{thm}
\begin{proof}Пусть $Y = \lambda f.(\lambda x.f\ (x\ x))\ (\lambda x.f\ (x\ x))$.
Тогда $R := Y\ N$:

$$Y\ N =_\beta (\lambda x.N\ ({\color{red}x}\ {\color{blue}x}))\ (\lambda x.N\ (x\ x)) =_\beta N\ ({\color{red}(\lambda x.N\ (x\ x)})\ ({\color{blue}\lambda x.N\ (x\ x)}))$$
\end{proof}
\end{frame}

\begin{frame}{Интуиционистское И.В. (натуральный, естественный вывод)}
\begin{itemize}
\item Формулы языка (секвенции) имеют вид: $\Gamma\vdash\alpha$.
Правила вывода: 
\begin{flushright}$\quad\quad\quad\infer[(\text{аннотация})]{\text{заключение}}{\text{посылка 1}\quad\quad\text{посылка 2}\quad\quad\dots}$\end{flushright}
\vspace{-0.7cm}
\item Аксиома:\\$\infer[\text{(акс.)}]{\Gamma,\alpha\vdash\alpha}{\vphantom{\Gamma}}$ 

\item Правила введения связок:\\$\infer{\Gamma\vdash\alpha\rightarrow\beta}{\Gamma,\alpha\vdash\beta}\quad\quad\infer{\Gamma\vdash\alpha\vee\beta}{\Gamma\vdash\alpha}$, $\infer{\Gamma\vdash\alpha\vee\beta}{\Gamma\vdash\beta}\quad\quad\infer{\Gamma\vdash\alpha\with\beta}{\Gamma\vdash\alpha\quad\quad\Gamma\vdash\beta}$

\item Правила удаления связок:\\$\infer{\Gamma\vdash\beta}{\Gamma\vdash\alpha\quad\Gamma\vdash\alpha\rightarrow\beta}\quad\quad\infer{\Gamma\vdash\gamma}{\Gamma\vdash\alpha\rightarrow\gamma\quad\Gamma\vdash\beta\rightarrow\gamma\quad\Gamma\vdash\alpha\vee\beta}$
 $\infer{\Gamma\vdash\alpha}{\Gamma\vdash\alpha\with\beta}\quad\quad\infer{\Gamma\vdash\beta}{\Gamma\vdash\alpha\with\beta}\quad\quad\infer{\Gamma\vdash\alpha}{\Gamma\vdash\bot}$
\item Пример доказательства:\vspace{-0.3cm}
$$\infer[(\text{введ}\with)]{A\with B\vdash B \with A}{\infer[(\text{удал}\with)]{A \with B \vdash B}{\infer[(\text{акс.})]{A \with B\vdash A \with B}{}}
                                           \quad\quad\infer[(\text{удал}\with)]{A \with B \vdash A}{\infer[(\text{акс.})]{A \with B\vdash A \with B}{}}}$$
\end{itemize}
\end{frame}

\begin{frame}{Эквивалентность натурального и гильбертовского выводов}
\begin{dfn}$$|\alpha|_\bot = \left\{\begin{array}{ll}X,&\alpha\equiv X\\|\sigma|_\bot\star|\tau|_\bot,& \alpha\equiv\sigma\star\tau\\
                                                     |\sigma|_\bot\rightarrow\bot,& \alpha\equiv\neg\sigma\end{array}\right.\quad\quad
             |\alpha|_\neg = \left\{\begin{array}{ll}X,&\alpha\equiv X\\|\sigma|_\neg\star|\tau|_\neg,& \alpha\equiv\sigma\star\tau\\A\with\neg A,& \alpha\equiv\bot\end{array}\right.
$$\end{dfn}
\begin{thm}\begin{enumerate}\item $\Gamma\vdash_n\alpha$ тогда и только тогда, когда $|\Gamma|_\neg\vdash_h|\alpha|_\neg$.
\item $\Gamma\vdash_h\alpha$ тогда и только тогда, когда $|\Gamma|_\bot\vdash_n|\alpha|_\bot$.
\end{enumerate}
\end{thm}
\begin{proof}Индукция по структуре\end{proof}
\end{frame}

\begin{frame}{Просто-типизированное лямбда-исчисление}
\begin{dfn}Импликационный фрагмент интуиционистской логики:

$$\infer{\Gamma,\varphi \vdash_\rightarrow \varphi}{} \quad\quad 
  \infer{\Gamma\vdash_\rightarrow\varphi\rightarrow\psi}{\Gamma,\varphi\vdash_\rightarrow\psi} \quad\quad 
  \infer{\Gamma\vdash_\rightarrow\psi}{\Gamma\vdash_\rightarrow\varphi\quad\quad\Gamma\vdash_\rightarrow\varphi\rightarrow\psi}$$
\end{dfn}

%\flushright <<Нет человека, равного Кари по храбрости>>\\
%Ярл Сигурд, Cага о Ньяле.
\vspace{-0.3cm}
\begin{thm}Если $\Gamma\vdash\alpha$, то $\Gamma\vdash_\rightarrow\alpha$.\end{thm}
\begin{proof}
Определим модель Крипке: \begin{itemize}
\item миры --- замкнутые множества формул: $\alpha\in\Gamma$ т.и.т.т. $\Gamma\vdash_\rightarrow\alpha$, 
\item порядок --- $(\subseteq)$,
\item $\Gamma\Vdash X$ т.и.т.т. $X\in\Gamma$.
\end{itemize}

%Мы покажем, что $\Gamma\Vdash\alpha$ т.и.т.т. $\Gamma\vdash_\rightarrow\alpha$.\pause

Из корректности моделей Крипке следует, что что если $\Gamma\vdash\alpha$, то $\Gamma\Vdash \alpha$.
Требуемое следует из того, что $\Gamma\Vdash \alpha$ влечёт $\Gamma\vdash_\rightarrow\alpha$.
\end{proof}
\end{frame}

\begin{frame}{$\Gamma\Vdash \alpha$ т.и.т.т. $\Gamma\vdash_\rightarrow\alpha$}
Индукция по структуре $\alpha$.
\begin{itemize}
\item $\alpha \equiv X$. Утверждение следует из определения;
\item $\alpha \equiv \varphi\rightarrow\psi$. 
\begin{itemize}

\item Пусть $\Gamma\Vdash \varphi\rightarrow\psi$. То есть, $\Gamma \subseteq \Delta$ и $\Delta \Vdash \varphi$
влечёт $\Delta \Vdash \psi$.
Возьмём $\Delta$ как замыкание $\Gamma\cup\{\varphi\}$. Значит, $\Delta\vdash_\rightarrow\varphi$
 и, по индукционному предположению, $\Delta\Vdash\varphi$.
Тогда $\Delta\Vdash\psi$. По индукционному предположению, $\Delta\vdash_\rightarrow\psi$. 
То есть, $\Gamma,\varphi\vdash_\rightarrow\psi$, откуда

$$\infer{\Gamma\vdash\varphi\rightarrow\psi}{\Gamma,\varphi\vdash\psi}$$
\item Пусть $\Gamma\vdash_\rightarrow\varphi\rightarrow\psi$. Проверим $\Gamma\Vdash\varphi\rightarrow\psi$.
Пусть $\Gamma \subseteq \Delta$ и пусть $\Delta\Vdash\varphi$. 

По индукционному предположению, $\varphi\in\Delta$.
То есть, $\Delta\vdash_\rightarrow\varphi$ и $\Delta\vdash_\rightarrow\varphi\rightarrow\psi$.
Тогда $$\infer{\Delta\vdash_\rightarrow\psi}{\Delta\vdash_\rightarrow\varphi\quad\Delta\vdash_\rightarrow\varphi\rightarrow\psi}$$

По индукционному предположению, $\Delta\Vdash\psi$, отчего $\Gamma\Vdash\varphi\rightarrow\psi$.
\end{itemize}
\end{itemize}
\end{frame}

\begin{frame}{Просто-типизированное лямбда-исчисление}
\begin{dfn}
Просто-типизированное лямбда-исчисление (по Карри). \pause Типы: $\tau ::= \alpha | (\tau\rightarrow\tau)$. \pause Язык: $\Gamma\vdash A:\varphi$
$$\infer[x \notin \Gamma]{\Gamma,x:\varphi \vdash x:\varphi}{} \quad\quad 
  \infer[x \notin \Gamma]{\Gamma\vdash \lambda x.A: \varphi\rightarrow\psi}{\Gamma,x:\varphi\vdash A:\psi} \quad\quad 
  \infer{\Gamma\vdash B A:\psi}{\Gamma\vdash A:\varphi\quad\quad\Gamma\vdash B:\varphi\rightarrow\psi}$$
\end{dfn}
\end{frame}

\begin{frame}{Пример: тип чёрчевских нумералов}
Пусть $\Gamma = f:\alpha\rightarrow\alpha, x: \alpha$

$$\infer[\lambda]{\vdash \lambda f.\lambda x.f\ (f\ x) : (\alpha\rightarrow\alpha)\rightarrow(\alpha\rightarrow\alpha)}{
  \infer[\lambda]{f: \alpha\rightarrow\alpha \vdash \lambda x.f\ (f\ x) : (\alpha\rightarrow\alpha)}{
    \infer[App]{{\color{blue}\{ f:\alpha\rightarrow\alpha, x: \alpha\}\ } \vdash f\ (f\ x): \alpha}{
      \infer[App]{\Gamma \vdash f\ x: \alpha}{
        \infer[Ax]{\Gamma \vdash x: \alpha}{}\quad\quad\infer[Ax]{\Gamma \vdash f: \alpha\rightarrow\alpha}{}
      }\quad\quad
      \infer[Ax]{\Gamma \vdash f: \alpha\rightarrow\alpha}{}
    }
  }
}$$
\end{frame}


\begin{frame}{Изоморфизм Карри-Ховарда}

\begin{tabular}{ll}
$\lambda$-исчисление & исчисление высказываний\\\hline
Выражение & доказательство\\
Тип выражения & высказывание\\
Тип функции & импликация\\
Упорядоченная пара & Конъюнкция\\
Алгебраический тип & Дизъюнкция\\
Необитаемый тип & Ложь
\end{tabular}

\end{frame}

\begin{frame}{Изоморфизм Карри-Ховарда: отрицание}

\begin{dfn}Ложь ($\bot$) --- необитаемый тип;
$\texttt{failwith/raise/throw} : \alpha\rightarrow\bot$; $\neg\varphi\equiv\varphi\rightarrow\bot$
 \end{dfn}

Например, контрапозиция:
$(\alpha\rightarrow\beta)\rightarrow(\neg\beta\rightarrow\neg\alpha)$

$$\infer[\lambda]{\lambda f^{\alpha\rightarrow\beta}.\lambda n^{\beta\rightarrow\bot}.\lambda a^\alpha.n\ (f\ a): (\alpha\rightarrow\beta)\rightarrow(\neg\beta\rightarrow\neg\alpha)}
        {\infer[\lambda]{f:\alpha\rightarrow\beta\vdash\lambda n^{\beta\rightarrow\bot}.\lambda a^\alpha.n\ (f\ a): \neg\beta\rightarrow\neg\alpha}
               {\infer[\lambda]{f:\alpha\rightarrow\beta,n:\beta\rightarrow\bot\vdash\lambda a^\alpha.n\ (f\ a): \neg\alpha}{
                       \infer[App]{f:\alpha\rightarrow\beta,n:\beta\rightarrow\bot, a:\alpha \vdash n\ (f\ a): \bot}{
                              \infer[App]{\Phi \vdash f\ a: \beta}{\infer[Ax]{\Phi \vdash a: \alpha}{}\quad\quad\infer[Ax]{\Phi \vdash f:\alpha\rightarrow\beta}{}}
    \quad\quad \infer[Ax]{\Phi \vdash n: \beta\rightarrow\bot}{}
               }
               }}}$$

Снятие двойного отрицания: $((\alpha\rightarrow\bot)\rightarrow\bot)\rightarrow\alpha$, то есть $\lambda f^{(\alpha\rightarrow\bot)\rightarrow\bot}.?: \alpha$.\\
$f$ угадывает, что передать $x: \alpha\rightarrow\bot$. Тогда надо по $f$ угадать, что передать $x$.
\end{frame}

\begin{frame}{Исчисление по Чёрчу и по Карри}
\begin{dfn}
Просто-типизированное лямбда-исчисление по Карри.
$$\infer[x \notin \Gamma]{\Gamma,x:\varphi \vdash x:\varphi}{} \quad\quad 
  \infer[x \notin \Gamma]{\Gamma\vdash \lambda x.A: \varphi\rightarrow\psi}{\Gamma,x:\varphi\vdash A:\psi} \quad\quad 
  \infer{\Gamma\vdash B A:\psi}{\Gamma\vdash A:\varphi\quad\quad\Gamma\vdash B:\varphi\rightarrow\psi}$$

Просто-типизированное лямбда-исчисление по Чёрчу. 
$$\infer[x \notin \Gamma]{\Gamma,x:\varphi \vdash x:\varphi}{} \quad\quad 
  \infer[x \notin \Gamma]{\Gamma\vdash \lambda x^{\color{blue}\varphi}.A: \varphi\rightarrow\psi}{\Gamma,x:\varphi\vdash A:\psi} \quad\quad 
  \infer{\Gamma\vdash B A:\psi}{\Gamma\vdash A:\varphi\quad\quad\Gamma\vdash B:\varphi\rightarrow\psi}$$
\end{dfn}\pause

\begin{exm}
\begin{tabular}{l|l}
По Карри & По Чёрчу\\\hline
$\lambda f.\lambda x.f\ (f\ x) : (\alpha\rightarrow\alpha)\rightarrow(\alpha\rightarrow\alpha)$ & $\lambda f^{\alpha\rightarrow\alpha}.\lambda x^\alpha.f\ (f\ x) : (\alpha\rightarrow\alpha)\rightarrow(\alpha\rightarrow\alpha)$\\\pause
$\lambda f.\lambda x.f\ (f\ x) : (\beta\rightarrow\beta)\rightarrow(\beta\rightarrow\beta)$ & $\lambda f^{\beta\rightarrow\beta}.\lambda x^\beta.f\ (f\ x) : (\beta\rightarrow\beta)\rightarrow(\beta\rightarrow\beta)$
\end{tabular}
\end{exm}

%Изоморофизм Карри-Ховарда:\\
%Типизированы по Чёрчу: Си, Паскаль, Джава, ...\\
%Типизированы по Карри: Окамль, Хаскель, ...
\end{frame}

\begin{frame}{Комбинаторы S,K}
\begin{dfn}Комбинатор --- лямбда-терм без свободных переменных
\end{dfn}
\begin{dfn}[исходная идея Моисея Шейнфинкеля, 1924]$S := \lambda x.\lambda y.\lambda z.x\ z\ (y\ z)$, $K := \lambda x.\lambda y.x$, $I := \lambda x.x$\\
(verSchmelzung, Konstanz --- исходно <<C>> у Шейнфинкеля, Identit\"at)
\end{dfn}

\begin{thm}Пусть $N$ --- некоторый замкнутый лямбда-терм. Тогда найдётся выражение $M$, состоящее из комбинаторов $S$,$K$, 
что $N =_\beta M$\end{thm}
%\begin{exm}$I =_\beta S\ K\ K$\end{exm}
\pause
\begin{exm}\vspace{-0.5cm}
$$I =_\beta S\ K\ K$$
\end{exm}
\begin{tabular}{ll}
$K := \lambda x^\alpha.\lambda y^\beta.x$ & $\alpha\rightarrow\beta\rightarrow\alpha$\\
$S := \lambda x^{\alpha\rightarrow\beta\rightarrow\gamma}.\lambda y^{\alpha\rightarrow\beta}.\lambda z^\alpha.x\ z\ (y\ z)$ & $(\alpha\rightarrow\beta\rightarrow\gamma)\rightarrow(\alpha\rightarrow\beta)\rightarrow\alpha\rightarrow\gamma$\\
\end{tabular}
\end{frame}

\end{document}
