\documentclass[aspectratio=169]{beamer}
\usepackage[utf8]{inputenc}
\usepackage[english,russian]{babel}
\usepackage{amssymb}
\usepackage{stmaryrd}
\usepackage{cmll}
\usepackage{xcolor}
\usepackage{proof}
\setbeamertemplate{navigation symbols}{}
%\usetheme{Warsaw}
\begin{document}

\newtheorem{axiom}{Аксиома}
\newtheorem{exmprus}{Пример}
\newtheorem{defrus}{Определение}
\newtheorem{lemmarus}{Лемма}
\newtheorem{thmrus}{Лемма}

\begin{frame}{}
\begin{center}{\Large Математическая логика}\\\itshape{КТ ИТМО, осень 2024 года}\end{center}
\end{frame}

\begin{frame}{Что такое правильное рассуждение?}
\begin{itemize}
\item Органон, Аристотель: 384-322 гг. до н.э.
\item Средневековье (<<фигуры>>, терминология).
\end{itemize}
\end{frame}

\begin{frame}{Математический анализ и его формализация}
%Джордж Беркли. Аналитик, или Рассуждение, адресованное неверующему математику. Опыт новой теории зрения.
\begin{itemize}
\item Ньютон, Лейбниц --- неформальная идея
\vspace{1cm}
\item Коши --- последовательности вместо бесконечно-малых, пределы
\vspace{1cm}
\item Вейерштрасс --- вещественные числа
\vspace{1cm}
\item Кантор --- теория множеств
\end{itemize}
\end{frame}

%\begin{frame}{}
%\item Формализация матанализа --- бесконечно-малые, последовательности, вещественные числа...\pause
%\item Теория множеств:
%
%$$0 := \varnothing; \quad 1 := \{ \varnothing \}; \quad n+1 := n \cup \{ n \} $$
%\end{itemize}
%\begin{frame}{Наивная теория множеств}
%$$0 := \varnothing; \quad 1 := \{ \varnothing \}; \quad n+1 := n \cup \{ n \} $$
%\end{frame}

\begin{frame}{Парадокс брадобрея, парадокс Рассела}

\begin{itemize}
\item На некотором острове живёт брадобрей, который бреет всех, кто не бреется
сам. Бреется ли сам брадобрей? \pause
\item Если
$$X = \{ x \ |\ x \notin x\}$$ \pause
то что можно сказать про
$$X \in X$$ \pause

\item \begin{itemize}
\item Пусть $X \in X$. Тогда $X : X \notin X$\pause
\item Пусть $X \notin X$. Тогда $X$ должен принадлежать $X$\pause
\end{itemize}

\item Не совсем парадокс: откуда мы знаем, что $X$ существует? \pause
Не совсем разрешение парадокса: а откуда мы знаем, что вещественные числа существуют?
\end{itemize}
\end{frame}

\begin{frame}{Программа Гильберта}
\begin{itemize}
\item Программа Гильберта: полностью формализовать математику, доказать непротиворечивость:
Neubegründung der Mathematik: Erste Mitteilung”, Abhandlungen aus dem Seminar der Hamburgischen Universität, 1: 157–177. Series of talks given at the University of Hamburg, July 25–27, 1921

\begin{itemize}
\item формализация всей математики;
\item доказательство полноты формализации (все факты могут быть доказаны в формализации);
\item непротиворечивость (невозможно вывести противоречие);
\item консервативность (любое доказательство о реальных объектах может быть сформулировано без использования идеальных объектов);
\item разрешимость (существует алгоритм, проверяющий истинность любого математического факта).
\end{itemize}

\item Теоремы Гёделя о неполноте формальной арифметики (1930) не дали реализовать её.
\end{itemize}
\end{frame}

\begin{frame}{Высказывание}
Высказывание --- это строка, сформированная по следующим правилам.\pause

\begin{itemize}
\item Атомарное высказывание --- пропозициональная переменная: $A, B', C_{1234}$ \pause

\item Составное высказывание: если $\alpha$ и $\beta$ --- высказывания, то высказываниями являются:
\begin{itemize}
\item Отрицание: $(\neg\alpha)$ \pause
\item Конъюнкция: $(\alpha\with\beta)\ \ \textit{или}\ \ (\alpha\wedge\beta)$ \pause
\item Дизъюнкция: $(\alpha\vee\beta)$ \pause
\item Импликация: $(\alpha\rightarrow\beta)\ \ \textit{или}\ \ (\alpha\supset\beta)$ \pause
\end{itemize}
\end{itemize}
Пример:
$$(((A\rightarrow B)\vee (B\rightarrow C)) \vee (C \rightarrow A))$$
\end{frame}

\begin{frame}{Соглашения о записи (метаязык)}
\begin{itemize}
\item Метапеременные:
$$\alpha, \beta, \gamma, \dots$$\pause

Если $\alpha$ --- высказывание, то $(\neg\alpha)$ --- высказывание\pause\vspace{0.3cm}

\item Метапеременные для пропозициональных переменных:
$$X, Y_{n}, Z'$$\pause

Пусть дана пропозициональная переменная $X$, тогда $(X\with(\neg X))$ --- высказывание
\end{itemize}

\end{frame}

\begin{frame}{Способы упростить запись}
\begin{itemize}
\item Приоритет связок: отрицание, конъюнкция, дизъюнкция, импликация \pause
\item Ассоциативность: левая для конъюнкции и дизъюнкции, правая для импликации \pause
\end{itemize}
Пример:
%((((A→B)&Q)∨(((¬B)→B)→C))∨(C→(C→A))) 
$$((((A\rightarrow B)\with Q)\vee(((\neg B)\rightarrow B) \rightarrow C)) \vee (C \rightarrow (C\rightarrow A)))$$
можем записать так:
$$(A\rightarrow B)\with Q\vee((\neg B\rightarrow B) \rightarrow C) \vee (C \rightarrow C\rightarrow A)$$
\end{frame}

\begin{frame}{Теория моделей}

Оценка высказываний: как их понимать?


\end{frame}

\iffalse
\begin{frame}{Пример: $((A \rightarrow B) \with (B \rightarrow C)) \rightarrow (A \rightarrow C)$}

%{Пример}
%<<Если это --- кот, {\color{blue}то} это --- млекопитающее {\color{red}и} 
%если это --- млекопитающее, {\color{blue}то} у него есть селезёнка,
%{\color{red}значит,} если это --- кот, {\color{blue}то} у этого есть селезёнка>>\pause

Как перевести $((A \rightarrow B) \with (B \rightarrow C)) \rightarrow (A \rightarrow C)$ на человеческий язык. \pause\vspace{0.3cm}

Пусть:

\begin{itemize}
\item A означает <<это --- кот>>;\pause
\item B означает <<это --- млекопитающее>>;\pause
\item C означает <<у этого есть селезёнка>>
\end{itemize}\pause\vspace{0.3cm}

Тогда:
\begin{center}
<<если это --- кот, {\color{blue}то} это --- млекопитающее>>\\\pause
{\color{red} и}\\\pause
<<если это --- млекопитающее, {\color{blue}то} у этого есть селезёнка>>\\\pause
{\color{red} значит}\\\pause
<<если это --- кот, {\color{blue}то} у этого есть селезёнка>>
\end{center}

\end{frame}
\fi

\begin{frame}{Неформальный пример: $(A \rightarrow B) \rightarrow (B \rightarrow A)$}

Давайте попробуем оценить высказывание $(A \rightarrow B) \rightarrow (B \rightarrow A)$.\pause

\vspace{0.5cm}

Если из $A$ следует $B$, то из $B$ следует $A$.\pause

\vspace{0.5cm}

Наверное, в общем случае это неверно. Например, пусть:

\begin{enumerate}
\item $A$ означает <<у меня есть кот>>;\pause
\item $B$ означает <<у меня есть животное>>.
\end{enumerate}\pause

\vspace{0.5cm}
Тогда:
\begin{enumerate}
\item $A \rightarrow B$ выполнена всегда;\pause
\item $B \rightarrow A$ может не выполняться: скажем, у меня есть собака, но нет кота.
%высказывание $A$ будет ложно, но высказывание $B$ по-прежнему истинно.
\end{enumerate}

\end{frame}


\begin{frame}{Оценка высказываний}

Высказывание $(A\rightarrow B)\rightarrow (B\rightarrow A)$ ложно, если, например:
\begin{itemize}
\item $A$ --- <<у меня есть кот>>;
\item $B$ --- <<у меня есть животное>>;
\item у меня есть собака, но нет кота.
\end{itemize}
\vspace{0.3cm}\pause
Иначе: $A$ ложно, $B$ истинно, тогда высказывание ложно.\pause\vspace{0.5cm}

Чтобы задать оценку высказываний:
\begin{itemize}
\item Зафиксируем множество истинностных значений $V = \{\textit{И},\textit{Л}\,\}$\pause
%\item Множество пропозициональных переменных $P$

\item Определим функцию оценки переменных (\emph{интерпретацию}) $f: P \rightarrow V$\\
(P --- множество пропозициональных переменных).
\end{itemize} \pause\vspace{0.3cm}

Если $\llbracket A \rrbracket = \textit{Л}$ и $\llbracket B \rrbracket = \textit{И}$,
то $\llbracket (A\rightarrow B)\rightarrow (B\rightarrow A) \rrbracket = \textit{Л}$

%\vspace{0.3cm}
%
%Оценка высказывания сопоставляет формуле истинностное значение:
%$$\llbracket A\rightarrow A \rrbracket = \textit{И}$$

\end{frame}

\begin{frame}{Указание функции оценки (метаязык)}

\begin{itemize}
\item Синтаксис для указания функции оценки переменных
$$\llbracket \alpha \rrbracket^{X_1 := v_1,\ \dots,\ X_n := v_n}$$
\item Это всё метаязык --- потому полагаемся на здравый смысл
$$\llbracket A \with B \with (C \rightarrow C) \rrbracket^{A := \textit{И},\ B := \llbracket \neg A \rrbracket}$$
\end{itemize}

\end{frame}

\begin{frame}{Оценим высказывания рекурсивно}
\begin{itemize}
\item Переменные $$\llbracket X \rrbracket = f(X)\quad\quad\quad \llbracket X \rrbracket^{X := a} = a$$ \pause\vspace{-0.3cm}
\item Отрицание $$\llbracket \neg \alpha \rrbracket = 
  \left\{\begin{array}{ll}\textit{Л},&\textit{если }\llbracket\alpha\rrbracket=\textit{И}\\
                        \textit{И},&\textit{иначе}\end{array}\right.$$ \pause\vspace{-0.1cm}
\item Конъюнкция $$\llbracket \alpha \with \beta \rrbracket = 
  \left\{\begin{array}{ll}\textit{И},&\textit{если }\llbracket\alpha\rrbracket=\llbracket\beta\rrbracket=\textit{И}\\ 
                        \textit{Л},&\textit{иначе}\end{array}\right.$$ \pause\vspace{-0.1cm}
\item Дизъюнкция $$\llbracket \alpha \vee \beta \rrbracket = 
  \left\{\begin{array}{ll}\textit{Л},&\textit{если }\llbracket\alpha\rrbracket=\llbracket\beta\rrbracket=\textit{Л}\\
                        \textit{И},&\textit{иначе}\end{array}\right.$$ \pause\vspace{-0.1cm}
\item Импликация $$\llbracket \alpha \rightarrow \beta \rrbracket = 
  \left\{\begin{array}{ll}\textit{Л},&\textit{если }\llbracket\alpha\rrbracket=\textit{И},\ \llbracket\beta\rrbracket=\textit{Л}\\
                        \textit{И},&\textit{иначе}\end{array}\right.$$
\end{itemize}
\end{frame}

\begin{frame}{Тавтологии}

Если $\alpha$ истинна при любой оценке переменных, то она \emph{общезначима} (является \emph{тавтологией}):
$$\models \alpha$$\pause


Выражение $A\rightarrow A$ --- тавтология. 
Переберём все возможные значения единственной переменной $A$:

$$
\begin{array}{l} \llbracket A\rightarrow A \rrbracket ^ {A := \textit{И}} = \textit{И} \\
 \llbracket A\rightarrow A \rrbracket ^ {A := \textit{Л}} = \textit{И} \end{array}
$$\pause

Выражение $A\rightarrow\neg A$ тавтологией не является:

$$\llbracket A\rightarrow\neg A \rrbracket ^ {A := \textit{И}} = \textit{Л}$$

\end{frame}

\begin{frame}{Ещё определения}
\begin{itemize}
\item Если $\alpha$ истинна при любой оценке переменных, при которой истинны 
высказывания $\gamma_1, \dots, \gamma_n$, будем говорить, что $\alpha$ --- \emph{следствие} этих высказываний:
$$\gamma_1, \dots, \gamma_n \models \alpha$$\pause
\item Истинна при какой-нибудь оценке --- \emph{выполнима}.\pause
\item Не истинна ни при какой оценке --- \emph{невыполнима}.\pause
\item Не истинна при какой-нибудь оценке --- \emph{опровержима}.
\end{itemize}
\end{frame}

\begin{frame}{Теория доказательств}

\begin{itemize}
\item Из чего состоит доказательство (неформально):
\begin{enumerate}
\item Аксиомы --- утверждения, от которых отталкиваемся.
\item Правила вывода --- способы делать умозаключения, переходить от одних утверждений к другим.
\end{enumerate}\pause

\item Давайте определим формально, что такое аксиомы и правила вывода, и затем дадим формальное 
определение доказательству как таковому.
\end{itemize}

\end{frame}

\begin{frame}{Схемы высказываний: определение}

\begin{defrus}[схема высказывания] Строка, строящаяся по правилам для построения высказываний, с одним отличием ---
вместо пропозициональных переменных можно указывать маленькие греческие буквы.\end{defrus}\pause

\emph{По-простому: схемы высказываний --- высказывания с метапеременными}\pause

\begin{exmprus}
\begin{itemize}
\item $(A \rightarrow \alpha) \vee (\beta \rightarrow B)$\pause
\item $(\alpha\rightarrow\beta)\rightarrow(\neg\beta\rightarrow\neg\alpha)$\pause
\item $A\vee B \with A$
\end{itemize}
\end{exmprus}

\end{frame}

\begin{frame}{Схемы высказываний: определение}

\begin{defrus}Будем говорить, что высказывание $\sigma$ строится (иначе: задаётся) по схеме $\textit{Ш}$, 
если существует такая замена метапеременных $\textit{ч}_1$, $\textit{ч}_2$, ..., $\textit{ч}_n$ 
в высказывании на какие-либо выражения $\varphi_1$, $\varphi_2$, ..., $\varphi_n$, 
что после её проведения получается высказывание $\sigma$:

$$\sigma = \textit{Ш}[\textit{ч}_1 := \varphi_1][\textit{ч}_2 := \varphi_2]...[\textit{ч}_n := \varphi_n]$$

Заметьте, здесь $\textit{ч}_i$ --- мета-метапеременные для метапеременных, а $\textit{Ш}$ --- мета-метапеременная для схем.
\end{defrus}
\end{frame}

\begin{frame}{Схемы высказываний: примеры}
%\begin{exmprus}[Схема высказывания]
Схема $$A \rightarrow \alpha \vee B \vee \alpha$$ задаёт, к примеру, следующие высказывания:
\begin{itemize}
\item $A \rightarrow X \vee B \vee X$, при $\alpha := X$.
\item $A \rightarrow (M\rightarrow N) \vee B \vee (M \rightarrow N)$, при $\alpha := M\rightarrow N$.
\end{itemize}

\vspace{0.5cm}\pause

и {\color{red} НЕ} задаёт следующие высказывания:
\begin{itemize}
\item {\color{red} $A \rightarrow {\color{red}X} \vee B \vee {\color{red}Y}$} --- все вхождения $\alpha$ должны заменяться одинаково во всём выражении.
\item { ${\color{red}\textbf{(}}{\color{gray}A \rightarrow (M \rightarrow N) \vee B \vee M}{\color{red}\textbf{)}} \color{gray}\rightarrow N$} --- структура скобок должна сохраняться.
\end{itemize}
%\end{exmprus}

\end{frame}

%Аксиомы --- некоторые выделенные высказывания\pause
%
%\vspace{0.3cm}
%
%Схема аксиом --- высказывание с метапеременными.\pause
%
%\vspace{0.3cm}
%
%Пример:
%$$\alpha\rightarrow\beta\rightarrow\alpha$$
%
%Заменим $\alpha$ и $\beta$ на выражения --- 
%получим аксиому.\pause
%
%$$\begin{array}{l}
%  \alpha := A\rightarrow B\\
%  \beta := A\with B\\
%  (A\rightarrow B)\rightarrow (A\with B)\rightarrow(A\rightarrow B)
%\end{array}$$

\begin{frame}{Аксиомы исчисления высказываний}

\begin{defrus}Назовём следующие схемы высказываний схемами аксиом исчисления высказываний:

\begin{tabular}{ll}
(1) & $\alpha \rightarrow \beta \rightarrow \alpha$ \\
(2) & $(\alpha \rightarrow \beta) \rightarrow (\alpha \rightarrow \beta \rightarrow \gamma) \rightarrow (\alpha \rightarrow \gamma)$ \\
(3) & $\alpha \rightarrow \beta \rightarrow \alpha \& \beta$\\
(4) & $\alpha \& \beta \rightarrow \alpha$\\
(5) & $\alpha \& \beta \rightarrow \beta$\\
(6) & $\alpha \rightarrow \alpha \vee \beta$\\
(7) & $\beta \rightarrow \alpha \vee \beta$\\
(8) & $(\alpha \rightarrow \gamma) \rightarrow (\beta \rightarrow \gamma) \rightarrow (\alpha \vee \beta \rightarrow \gamma)$\\
(9) & $(\alpha \rightarrow \beta) \rightarrow (\alpha \rightarrow \neg \beta) \rightarrow \neg \alpha$\\
(10) & $\neg \neg \alpha \rightarrow \alpha$
\end{tabular}

Все высказывания, которые задаются схемами аксиом, назовём аксиомами исчисления высказываний.
\end{defrus}
\end{frame}

\begin{frame}{Правило вывода Modus Ponens}
Первый, упомянувший правило --- Теофраст (древнегреческий философ, IV-III век до н.э.).\pause

\vspace{1cm}

Переход по следствию: <<сейчас сентябрь; если сейчас сентябрь, то сейчас осень; следовательно, сейчас осень>>.\pause

\vspace{1cm}

Если имеет место $\alpha$ и $\alpha\rightarrow\beta$, то имеет место $\beta$.

$$\infer{\beta}{\alpha & \alpha\rightarrow\beta}$$
\end{frame}

\begin{frame}{Доказательство}
\begin{defrus}[доказательство в исчислении высказываний]
Доказательством (выводом) назовём конечную последовательность высказываний $\delta_1, \delta_2, \dots, \delta_n$, \pause
причём каждое $\delta_i$ либо:
\begin{itemize}
\item является аксиомой --- существует замена метапеременных для какой-либо схемы аксиом, позволяющая получить
формулу $\delta_i$, либо\pause
\item получается из $\delta_1,\dots,\delta_{i-1}$ по правилу Modus Ponens --- существуют такие индексы $j < i$ и $k < i$,
что $\delta_k \equiv \delta_j\rightarrow\delta_i$.
\end{itemize}\end{defrus}\pause

Пример:\vspace{0.3cm}

$A \rightarrow (A \rightarrow A)$,

$(A \rightarrow (A \rightarrow A)) \rightarrow 
  (A \rightarrow ((A \rightarrow A) \rightarrow A)) \rightarrow
  (A \rightarrow A)$,

$(A \rightarrow ((A \rightarrow A) \rightarrow A)) \rightarrow
  (A \rightarrow A)$,

$A \rightarrow ((A \rightarrow A) \rightarrow A)$,

$A \rightarrow A$\end{frame}

\begin{frame}{Доказательство подробнее}
Почему это доказательство? То же подробнее:\pause\vspace{0.2cm}

\begin{tabular}{lll}
(1) & $A \rightarrow (A \rightarrow A)$&Сх. акс. 1\\
    & {\color{cyan}$\alpha \rightarrow \beta \rightarrow \alpha\ [\alpha, \beta := A]$}

\vspace{0.2cm}\\\pause

(2) & $(A \rightarrow (A \rightarrow A)) \rightarrow 
  (A \rightarrow ((A \rightarrow A) \rightarrow A)) \rightarrow
  (A \rightarrow A)$&Сх. акс. 2\\
    & {\color{cyan}$(\alpha \rightarrow \beta) \rightarrow (\alpha \rightarrow \beta \rightarrow \gamma) 
       \rightarrow (\alpha \rightarrow \gamma)\ [\alpha, \gamma := A; \beta := A\rightarrow A]$}

\vspace{0.2cm}\\\pause

(3) & $(A \rightarrow ((A \rightarrow A) \rightarrow A)) \rightarrow
  (A \rightarrow A)$&M.P. 1,2\\
    \multicolumn{3}{c}{\color{cyan} $\infer{(A \rightarrow ((A \rightarrow A) \rightarrow A)) \rightarrow
  (A \rightarrow A)}{A \rightarrow (A \rightarrow A) & (A \rightarrow (A \rightarrow A)) \rightarrow 
  (A \rightarrow ((A \rightarrow A) \rightarrow A)) \rightarrow
  (A \rightarrow A)}$}

\vspace{0.2cm}\\\pause

(4) & $A \rightarrow ((A \rightarrow A) \rightarrow A)$ & Сх. акс. 1\\
    & {\color{cyan}$\alpha \rightarrow \beta \rightarrow \alpha\ [\alpha := A, \beta := A\rightarrow A]$}

\vspace{0.2cm}\\\pause

(5) & $A \rightarrow A$ & M.P. 4,3\\
    & {\color{cyan} $\infer{A \rightarrow A}{A \rightarrow ((A \rightarrow A) \rightarrow A) & 
       (A \rightarrow ((A \rightarrow A) \rightarrow A)) \rightarrow (A \rightarrow A)}$}

\vspace{0.2cm}
\end{tabular}
\end{frame}

\begin{frame}{Дополнительные определения}

\begin{defrus}[доказательство формулы $\alpha$] 
--- такое доказательство (вывод) $\delta_1, \delta_2, \dots, \delta_n$,
что $\alpha\equiv\delta_n$.

Формула $\alpha$ доказуема (выводима), если существует её доказательство. Обозначение:
$$\vdash \alpha$$\end{defrus}\pause

\begin{defrus}[вывод формулы $\alpha$ из гипотез $\gamma_1,\dots,\gamma_k$]
--- такая последовательность
$\delta_1,\dots,\delta_n$, причём каждое $\delta_i$ либо:
\begin{itemize}
\item является аксиомой;
\item либо получается по правилу Modus Ponens из предыдущих;
\item либо является одной из гипотез: существует $t: \delta_i \equiv \gamma_t$.
\end{itemize}

Формула $\alpha$ выводима из гипотез $\gamma_1,\dots,\gamma_k$, если существует её вывод. Обозначение:
$$\gamma_1,\dots,\gamma_k\vdash\alpha$$\end{defrus}

\end{frame}

\begin{frame}{Корректность и полнота}
\begin{defrus}[корректность теории]
Теория корректна, если любое доказуемое в ней утверждение общезначимо.
То есть, $\vdash\alpha$ влечёт $\models\alpha$.
\end{defrus}

\begin{defrus}[полнота теории]
Теория полна, если любое общезначимое в ней утверждение доказуемо.
То есть, $\models\alpha$ влечёт $\vdash\alpha$.
\end{defrus}
\end{frame}

\begin{frame}{Корректность исчисления высказываний}
\begin{thmrus}[корректность]
Если $\vdash\alpha$, то $\models\alpha$
\end{thmrus}

\begin{proof}
Индукция по длине вывода $n$.
Для каждого высказывания $\delta_n$ из вывода разбор случаев:
\begin{enumerate}
\item Аксиома --- убедиться, что все аксиомы общезначимы.
\item Modus Ponens $j$, $k$ --- убедиться, что если $\models\delta_j$ и 
$\models\delta_j\rightarrow\delta_n$, то $\models\delta_n$.
\end{enumerate}
\end{proof}
\end{frame}

\begin{frame}{Общезначимость схемы аксиом №9}
Общезначимость схемы аксиом --- истинность каждой аксиомы, задаваемой данной схемой, при любой оценке:
$$\llbracket(\alpha\rightarrow\beta)\rightarrow(\alpha\rightarrow\neg\beta)\rightarrow\neg\alpha\rrbracket
   = \textnormal{И}$$

Построим таблицу истинности формулы в зависимости от оценки $\alpha$ и $\beta$:
\vspace{0.3cm}

{\footnotesize
\begin{tabular}{cc|ccccc}
$\llbracket\alpha\rrbracket$ & $\llbracket\beta\rrbracket$ & $\llbracket\neg\alpha\rrbracket$ & 
    $\llbracket\alpha\rightarrow\beta\rrbracket$ & $\llbracket\alpha\rightarrow\neg\beta\rrbracket$ 
  & $\llbracket(\alpha\rightarrow\neg\beta)\rightarrow\neg\alpha\rrbracket$ & 
    $\llbracket(\alpha\rightarrow\beta)\rightarrow(\alpha\rightarrow\neg\beta)\rightarrow\neg\alpha\rrbracket$\\
\hline
  Л & Л & И & И & И & И & И\\
  Л & И & И & И & И & И & И\\
  И & Л & Л & Л & И & Л & И\\
  И & И & Л & И & Л & И & И
\end{tabular}}

\end{frame}

\begin{frame}{Общезначимость заключения правила Modus Ponens}
Пусть в выводе есть формулы $\delta_j$, $\delta_k = \delta_j\rightarrow\delta_n$, $\delta_n$ (причём
$j < n$ и $k < n$).\vspace{0.3cm}\pause


Фиксируем какую-нибудь оценку. 
По индукционному предположению, $\delta_j$ и $\delta_j\rightarrow\delta_n$ общезначимы.
Поэтому при данной оценке $\llbracket\delta_j\rrbracket = \textnormal{И}$ и
$\llbracket\delta_j\rightarrow\delta_n\rrbracket = \textnormal{И}$.\vspace{0.3cm}\pause

Построим таблицу истинности для импликации:

\begin{center}\begin{tabular}{ccc}
$\llbracket\delta_j\rrbracket$ &$\llbracket\delta_n\rrbracket$ & $\llbracket\delta_j\rightarrow\delta_n\rrbracket$\\
\hline
Л & Л & И \\
Л & И & И \\
И & Л & Л \\
И & И & И
\end{tabular}\end{center}\pause

Из таблицы видно, что $\llbracket\delta_n\rrbracket = \textnormal{Л}$ только если 
$\llbracket\delta_j\rightarrow\delta_n\rrbracket = \textnormal{Л}$ или 
$\llbracket\delta_j\rrbracket = \textnormal{Л}$. Значит, это невозможно, и
$\llbracket\delta_n\rrbracket = \textnormal{И}$

\end{frame}

\end{document}